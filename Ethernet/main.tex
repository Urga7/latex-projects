\documentclass{article}
\usepackage[slovene]{babel}
\usepackage[margin=2cm,bottom=3cm,foot=1.5cm]{geometry}
\usepackage{graphicx}
\usepackage{pgfplots}
\usepackage{float}
\usepackage{tabularx}
\usepackage{multirow}
\usepackage{booktabs}
\usepackage{array}
\pgfplotsset{compat=1.18}
\usepackage{amsmath}

\renewcommand{\figurename}{Slika}

\begin{document}

\title{Modeliranje protokola Ethernet}
\author{Luka Kosec Mikić in Urban Gajšek}
\date{\today}

\maketitle

\begin{center}
    \includegraphics[width=0.5\textwidth]{figures/univerza-v-ljubljani.png}
\end{center}

\newpage

\section{Uvod}

Ethernet je ena izmed najpogosteje uporabljenih tehnologij za povezovanje naprav v
lokalna omrežja (LAN). Razvit je bil v 70-ih letih prejšnjega stoletja in od takrat
postaja standard za žična omrežja zaradi svoje preprostosti, zanesljivosti in nizkih
stroškov implementacije. Temelji na uporabi podatkovnih okvirjev (ang. frames),
ki se prenašajo po skupnem komunikacijskem mediju, pri čemer lahko omrežne naprave
komunicirajo med seboj prek stikal in razdelilcev.

Z razvojem omrežij so se pojavile različne različice Ethernet tehnologije,
ki podpirajo različne hitrosti prenosa podatkov, od prvotnih 10 Mbps pa vse do
sodobnih hitrosti 1 Gbps, 10 Gbps in več. Ethernet omogoča uporabo različnih topologij
omrežja, kot so zvezda, vodilo in obroč, ter podpira tako pol- kot tudi polni duplex
način prenosa podatkov. Zaradi svoje široke uporabe je pomembno razumeti delovanje te
tehnologije in sposobnost modeliranja različnih omrežnih scenarijev.

V tej seminarski nalogi bomo s pomočjo ogrodja INET v simulacijskem orodju OMNeT++
modelirali različne primere Ethernet omrežij. S tem bomo pridobili vpogled v delovanje
Ethernet omrežij, analizirali njihovo zmogljivost ter preučili vpliv različnih parametrov
na delovanje omrežij. V uvodnem delu bomo predstavili osnove modeliranja Ethernet omrežij
v ogrodju INET ter opisali nekaj osnovnih primerov omrežij, ki bodo služili kot osnova za
nadaljnjo analizo.

\newpage
\tableofcontents
\newpage


\end{document}